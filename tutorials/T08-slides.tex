
\documentclass[letterpaper,hide notes,xcolor={table,svgnames},pdftex,10pt]{beamer}
\def\showexamples{t}


%\usepackage[svgnames]{xcolor}

%% Demo talk
%\documentclass[letterpaper,notes=show]{beamer}

\usecolortheme{crane}
\setbeamertemplate{navigation symbols}{}

\usetheme{MyPittsburgh}
%\usetheme{Frankfurt}

%\usepackage{tipa}

\usepackage{hyperref}
\usepackage{graphicx,xspace}
\usepackage[normalem]{ulem}
\usepackage{multicol}
\usepackage{amsmath,amssymb,amsthm,graphicx,xspace}
\newcommand\SF[1]{$\bigstar$\footnote{SF: #1}}

\usepackage[default]{sourcesanspro}
\usepackage[T1]{fontenc}

\newcounter{tmpnumSlide}
\newcounter{tmpnumNote}

% old question code
%\newcommand\question[1]{{$\bigstar$ \small \onlySlide{2}{#1}}}
% \newcommand\nquestion[1]{\ifdefined \presentationonly \textcircled{?} \fi \note{\par{\Large \textbf{?}} #1}}
% \newcommand\nanswer[1]{\note{\par{\Large \textbf{A}} #1}}


 \newcommand\mnote[1]{%
   \addtocounter{tmpnumSlide}{1}
   \ifdefined\showcues {~\tiny\fbox{\arabic{tmpnumSlide}}}\fi
   \note{\setlength{\parskip}{1ex}\addtocounter{tmpnumNote}{1}\textbf{\Large \arabic{tmpnumNote}:} {#1\par}}}

\newcommand\mmnote[1]{\note{\setlength{\parskip}{1ex}#1\par}}

%\newcommand\mnote[2][]{\ifdefined\handoutwithnotes {~\tiny\fbox{#1}}\fi
% \note{\setlength{\parskip}{1ex}\textbf{\Large #1:} #2\par}}

%\newcommand\mnote[2][]{{\tiny\fbox{#1}} \note{\setlength{\parskip}{1ex}\textbf{\Large #1:} #2\par}}

\newcommand\mquestion[2]{{~\color{red}\fbox{?}}\note{\setlength{\parskip}{1ex}\par{\Large \textbf{?}} #1} \note{\setlength{\parskip}{1ex}\par{\Large \textbf{A}} #2\par}\ifdefined \presentationonly \pause \fi}

\newcommand\blackboard[1]{%
\ifdefined   \showblackboard
  {#1}
  \else {\begin{center} \fbox{\colorbox{blue!30}{%
         \begin{minipage}{.95\linewidth}%
           \hspace{\stretch{1}} Some space intentionally left blank; done at the blackboard.%
         \end{minipage}}}\end{center}}%
         \fi%
}



%\newcommand\q{\tikz \node[thick,color=black,shape=circle]{?};}
%\newcommand\q{\ifdefined \presentationonly \textcircled{?} \fi}

\usepackage{listings}
\lstset{%
  keywordstyle=\bfseries,
  aboveskip=15pt,
  belowskip=15pt,
  captionpos=b,
  identifierstyle=\ttfamily,
  escapeinside={(*@}{@*)},
  stringstyle=\ttfamiliy,
  frame=lines,
  numbers=left, basicstyle=\scriptsize, numberstyle=\tiny, stepnumber=0, numbersep=2pt}

\usepackage{siunitx}
\newcommand\sius[1]{\num[group-separator = {,}]{#1}\si{\micro\second}}
\newcommand\sims[1]{\num[group-separator = {,}]{#1}\si{\milli\second}}
\newcommand\sins[1]{\num[group-separator = {,}]{#1}\si{\nano\second}}
\sisetup{group-separator = {,}, group-digits = true}

%% -------------------- tikz --------------------
\usepackage{tikz}
\usetikzlibrary{positioning}
\usetikzlibrary{arrows,backgrounds,automata,decorations.shapes,decorations.pathmorphing,decorations.markings,decorations.text}

\tikzstyle{place}=[circle,draw=blue!50,fill=blue!20,thick, inner sep=0pt,minimum size=6mm]
\tikzstyle{transition}=[rectangle,draw=black!50,fill=black!20,thick, inner sep=0pt,minimum size=4mm]

\tikzstyle{block}=[rectangle,draw=black, thick, inner sep=5pt]
\tikzstyle{bullet}=[circle,draw=black, fill=black, thin, inner sep=2pt]

\tikzstyle{pre}=[<-,shorten <=1pt,>=stealth',semithick]
\tikzstyle{post}=[->,shorten >=1pt,>=stealth',semithick]
\tikzstyle{bi}=[<->,shorten >=1pt,shorten <=1pt, >=stealth',semithick]

\tikzstyle{mut}=[-,>=stealth',semithick]

\tikzstyle{treereset}=[dashed,->, shorten >=1pt,>=stealth',thin]

\usepackage{ifmtarg}
\usepackage{xifthen}
\makeatletter
% new counter to now which frame it is within the sequence
\newcounter{multiframecounter}
% initialize buffer for previously used frame title
\gdef\lastframetitle{\textit{undefined}}
% new environment for a multi-frame
\newenvironment{multiframe}[1][]{%
\ifthenelse{\isempty{#1}}{%
% if no frame title was set via optional parameter,
% only increase sequence counter by 1
\addtocounter{multiframecounter}{1}%
}{%
% new frame title has been provided, thus
% reset sequence counter to 1 and buffer frame title for later use
\setcounter{multiframecounter}{1}%
\gdef\lastframetitle{#1}%
}%
% start conventional frame environment and
% automatically set frame title followed by sequence counter
\begin{frame}%
\frametitle{\lastframetitle~{\normalfont(\arabic{multiframecounter})}}%
}{%
\end{frame}%
}
\makeatother

\makeatletter
\newdimen\tu@tmpa%
\newdimen\ydiffl%
\newdimen\xdiffl%
\newcommand\ydiff[2]{%
    \coordinate (tmpnamea) at (#1);%
    \coordinate (tmpnameb) at (#2);%
    \pgfextracty{\tu@tmpa}{\pgfpointanchor{tmpnamea}{center}}%
    \pgfextracty{\ydiffl}{\pgfpointanchor{tmpnameb}{center}}%
    \advance\ydiffl by -\tu@tmpa%
}
\newcommand\xdiff[2]{%
    \coordinate (tmpnamea) at (#1);%
    \coordinate (tmpnameb) at (#2);%
    \pgfextractx{\tu@tmpa}{\pgfpointanchor{tmpnamea}{center}}%
    \pgfextractx{\xdiffl}{\pgfpointanchor{tmpnameb}{center}}%
    \advance\xdiffl by -\tu@tmpa%
}
\makeatother
\newcommand{\copyrightbox}[3][r]{%
\begin{tikzpicture}%
\node[inner sep=0pt,minimum size=2em](ciimage){#2};
\usefont{OT1}{phv}{n}{n}\fontsize{4}{4}\selectfont
\ydiff{ciimage.south}{ciimage.north}
\xdiff{ciimage.west}{ciimage.east}
\ifthenelse{\equal{#1}{r}}{%
\node[inner sep=0pt,right=1ex of ciimage.south east,anchor=north west,rotate=90]%
{\raggedleft\color{black!50}\parbox{\the\ydiffl}{\raggedright{}#3}};%
}{%
\ifthenelse{\equal{#1}{l}}{%
\node[inner sep=0pt,right=1ex of ciimage.south west,anchor=south west,rotate=90]%
{\raggedleft\color{black!50}\parbox{\the\ydiffl}{\raggedright{}#3}};%
}{%
\node[inner sep=0pt,below=1ex of ciimage.south west,anchor=north west]%
{\raggedleft\color{black!50}\parbox{\the\xdiffl}{\raggedright{}#3}};%
}
}
\end{tikzpicture}
}


%% --------------------

%\usepackage[excludeor]{everyhook}
%\PushPreHook{par}{\setbox0=\lastbox\llap{MUH}}\box0}

%\vspace*{\stretch{1}

%\setbox0=\lastbox \llap{\textbullet\enskip}\box0}

\setlength{\parskip}{\fill}

\newcommand\noskips{\setlength{\parskip}{1ex}}
\newcommand\doskips{\setlength{\parskip}{\fill}}

\newcommand\xx{\par\vspace*{\stretch{1}}\par}
\newcommand\xxs{\par\vspace*{2ex}\par}
\newcommand\tuple[1]{\langle #1 \rangle}
\newcommand\code[1]{{\sf \footnotesize #1}}
\newcommand\ex[1]{\uline{Example:} \ifdefined \presentationonly \pause \fi
  \ifdefined\showexamples#1\xspace\else{\uline{\hspace*{2cm}}}\fi}

\newcommand\ceil[1]{\lceil #1 \rceil}


\AtBeginSection[]
{
   \begin{frame}
       \frametitle{Outline}
       \tableofcontents[currentsection]
   \end{frame}
}



\pgfdeclarelayer{edgelayer}
\pgfdeclarelayer{nodelayer}
\pgfsetlayers{edgelayer,nodelayer,main}

\tikzstyle{none}=[inner sep=0pt]
\tikzstyle{rn}=[circle,fill=Red,draw=Black,line width=0.8 pt]
\tikzstyle{gn}=[circle,fill=Lime,draw=Black,line width=0.8 pt]
\tikzstyle{yn}=[circle,fill=Yellow,draw=Black,line width=0.8 pt]
\tikzstyle{empty}=[circle,fill=White,draw=Black]
\tikzstyle{bw} = [rectangle, draw, fill=blue!20, 
    text width=4em, text centered, rounded corners, minimum height=2em]
    
    \newcommand{\CcNote}[1]{% longname
	This work is licensed under the \textit{Creative Commons #1 3.0 License}.%
}
\newcommand{\CcImageBy}[1]{%
	\includegraphics[scale=#1]{creative_commons/cc_by_30.pdf}%
}
\newcommand{\CcImageSa}[1]{%
	\includegraphics[scale=#1]{creative_commons/cc_sa_30.pdf}%
}
\newcommand{\CcImageNc}[1]{%
	\includegraphics[scale=#1]{creative_commons/cc_nc_30.pdf}%
}
\newcommand{\CcGroupBySa}[2]{% zoom, gap
	\CcImageBy{#1}\hspace*{#2}\CcImageNc{#1}\hspace*{#2}\CcImageSa{#1}%
}
\newcommand{\CcLongnameByNcSa}{Attribution-NonCommercial-ShareAlike}

\newenvironment{changemargin}[1]{% 
  \begin{list}{}{% 
    \setlength{\topsep}{0pt}% 
    \setlength{\leftmargin}{#1}% 
    \setlength{\rightmargin}{1em}
    \setlength{\listparindent}{\parindent}% 
    \setlength{\itemindent}{\parindent}% 
    \setlength{\parsep}{\parskip}% 
  }% 
  \item[]}{\end{list}} 




\def\ojoin{\setbox0=\hbox{$\bowtie$}%
  \rule[-.02ex]{.25em}{.4pt}\llap{\rule[\ht0]{.25em}{.4pt}}}
\def\leftouterjoin{\mathbin{\ojoin\mkern-5.8mu\bowtie}}

\title{Tutorial 8 --- Transactions and Recovery }

\author{Richard Wong \\ \small \texttt{rk2wong@edu.uwaterloo.ca}}
\institute{Department of Electrical and Computer Engineering \\
  University of Waterloo}
\date{\today}


\begin{document}

\begin{frame}
  \titlepage

\end{frame}


\begin{frame}
\frametitle{Exercise 8-1}

Is the following transaction schedule \textit{recoverable}?

Can we make a conflict-equivalent schedule that is \textit{cascadeless}?

\begin{center}
\begin{tabular}{ c c c c c c c c c c }
  \hline
  T1 & $r_x$ & & $r_y$ & & & & & & c \\
  \hline
  T2 & & $w_y$ & & & $r_x$ & & & c & \\
  \hline
  T3 & & & & $w_x$ & & $r_x$ & c & & \\
  \hline
\end{tabular}
\end{center}

\end{frame}


\begin{frame}
\frametitle{Exercise 8-1 Solution (1/2)}

In the context of recoverability, a transaction T \textbf{depends on} a transaction S if T reads a value that S has written to previously.

For a schedule to be recoverable, for any pair of transactions S and T, if T depends on S, then S must commit \textit{before} T does.

In the given schedule, that means:

\begin{itemize}
  \item T3 must commit before T2 does, and
  \item T2 must commit before T1 does.
\end{itemize}

\end{frame}


\begin{frame}
\frametitle{Exercise 8-1 Solution (2/2)}

For a schedule to be recoverable, for any pair of transactions S and T, if T depends on S, then S must commit \textit{before} T reads a dependent value.

The following schedule is conflict-equivalent to the original, and is also cascadeless.

\begin{center}
\begin{tabular}{ c c c c c c c c c c }
  \hline
  T1 & $r_x$ & & & & & & & $r_y$ & c \\
  \hline
  T2 & & $w_y$ & & & & $r_x$ & c & & \\
  \hline
  T3 & & & $w_x$ & $r_x$ & c & & & & \\
  \hline
\end{tabular}
\end{center}

Note: In the case of an abort, only one transaction will need to roll back and the others can run; this does not preserve equivalence to the original schedule, but can still be consistent.

\end{frame}


\begin{frame}
\frametitle{Exercise 8-2}

What is the weakest isolation guarantee available in SQL?

When would an developer want to use a weaker isolation level in their application?

\end{frame}


\begin{frame}
\frametitle{Exercise 8-2 Solution}

The weakest isolation level provided by SQL is \textbf{read-uncommitted}.

This level guarantees \textit{no dirty writes}: no writes on top of uncommitted writes by other transactions.

Weaker isolation affords a greater degree of concurrency, for a potential boost in throughput. Use if the possibly-resulting inconsistencies are irrelevant to the application, or they are otherwise worth dealing with.

\end{frame}


\begin{frame}
\frametitle{Exercise 8-3}

Some DMBSes use snapshot isolation to implement \textit{serializable}-level isolation. It works most of the time, but has failure cases.

How can snapshot isolation fail to create a serializable schedule, and what should happen when it creates a non-serializable schedule?

\end{frame}


\begin{frame}
\frametitle{Exercise 8-3 Solution}

In \textbf{snapshot isolation}, transactions operate on a snapshot of the DB that looks as it did when the transaction started.

A consistency check (for dirty writes, essentially) is performed right before an attempted commit.

Snapshot isolation can result in non-serializable schedules because no additional work is done to ensure serializability as the schedule is generated. The consistency check at the end will abort non-serializable schedules.

\end{frame}


\begin{frame}
\frametitle{Exercise 8-4}

Show that 2PL can create schedules that result in deadlock.

What can we do to \textbf{prevent} deadlock?

\end{frame}


\begin{frame}
\frametitle{Exercise 8-4 Solution (1/2)}

Recall that deadlock requires four conditions to be simultaneously true:

\begin{enumerate}
  \item mutual exclusion,
  \item hold-and-wait,
  \item no pre-emption, and
  \item circular wait.
\end{enumerate}

A 2PL schedule that results in deadlock, supposing T1 wants to read resources $a$ and $b$, and T2 wants to write to those same resources:

\begin{center}
\begin{tabular}{ c c c c c }
  \hline
  T1 & $s_a$ &      &      & $s_b$ \\
  \hline
  T2 &      & $x_b$ & $x_a$ &      \\
  \hline
\end{tabular}
\end{center}

\end{frame}


\begin{frame}
\frametitle{Exercise 8-4 Solution (2/2)}

Note that in 2PL, transactions can run into deadlock only during their growing phase.

We can prevent deadlock by having a smarter lock manager. There are several ways to prevent deadlock. The following are some things that the lock manager can do:

\begin{itemize}
  \item Use a deadlock detection algorithm (e.g. Banker's, graph algorithms) to determine whether granting a lock might lead to deadlock, and make the caller wait or abort if so. This can be expensive.
  \item Enforce a partial ordering (could be encoded with a tree) for lock acquisition to remove the possibility of circular wait.
\end{itemize}

\end{frame}


\begin{frame}
\frametitle{Exercise 8-5}

Show that 2PL can create recoverable schedules with cascading rollbacks.

What variant of 2PL creates cascadeless schedules?

\end{frame}


\begin{frame}
\frametitle{Exercise 8-5 Solution}

\begin{center}
\begin{tabular}{ c c c c c c c c c c c c c }
  \hline
  T1 & $x_a$ & $w_a$ & $u_a$ &      &      &      &      &      &      &      &      & abort \\
  \hline
  T2 &      &      &      & $s_a$ & $r_a$ & $x_a$ & $w_a$ & $u_a$ &      &      &      &      \\
  \hline
  T3 &      &      &      &      &      &      &      &      & $s_a$ & $r_a$ & $u_a$ &      \\
  \hline
\end{tabular}
\end{center}

\end{frame}


\begin{frame}
\frametitle{Exercise 8-6}

Supposing that instead handling deadlock with an avoidance/prevention strategy, we try to detect and recover. How do we recover from a deadlock?

\end{frame}


\begin{frame}
\frametitle{Exercise 8-6 Solution}

Recovering from a deadlock involves rolling back transactions until the circular wait is removed.

We choose a transaction to roll back using some heuristic that takes starvation into account, then roll back some or all of the transaction.

\end{frame}


\end{document}
