\documentclass[letterpaper,10pt]{article}

\usepackage{titling}
\usepackage{listings}
\usepackage{url}
\usepackage{setspace}
\usepackage{subfig}
\usepackage{sectsty}
\usepackage{pdfpages}
\usepackage{colortbl}
\usepackage{multirow}
\usepackage{multicol}
\usepackage{relsize}
\usepackage{amsmath}
\usepackage{wasysym}
\usepackage{fancyvrb}
\usepackage{amssymb}
\usepackage{ifsym}
\usepackage{amsmath,amssymb,amsthm,graphicx,xspace}
\usepackage[titlenotnumbered,noend,noline]{algorithm2e}
\usepackage[compact]{titlesec}
\usepackage{XCharter}
\usepackage[T1]{fontenc}
\usepackage{tikz}
\usetikzlibrary{arrows,automata,shapes,trees,matrix,chains,scopes,positioning,calc}
\tikzstyle{block} = [rectangle, draw, fill=blue!20, 
    text width=2.5em, text centered, rounded corners, minimum height=2em]
\tikzstyle{bw} = [rectangle, draw, fill=blue!20, 
    text width=4em, text centered, rounded corners, minimum height=2em]

\definecolor{namerow}{cmyk}{.40,.40,.40,.40}
\definecolor{namecol}{cmyk}{.40,.40,.40,.40}

\let\LaTeXtitle\title
\renewcommand{\title}[1]{\LaTeXtitle{\textsf{#1}}}


\newcommand{\handout}[5]{
  \noindent
  \begin{center}
  \framebox{
    \vbox{
      \hbox to 5.78in { {\bf ECE356: Database Systems } \hfill #2 }
      \vspace{4mm}
      \hbox to 5.78in { {\Large \hfill #4  \hfill} }
      \vspace{2mm}
      \hbox to 5.78in { {\em #3 \hfill} }
    }
  }
  \end{center}
  \vspace*{4mm}
}

\newcommand{\lecture}[3]{\handout{#1}{#2}{#3}{Lecture #1}}
\newcommand{\tuple}[1]{\ensuremath{\left\langle #1 \right\rangle}\xspace}

\addtolength{\oddsidemargin}{-1.000in}
\addtolength{\evensidemargin}{-0.500in}
\addtolength{\textwidth}{2.0in}
\addtolength{\topmargin}{-1.000in}
\addtolength{\textheight}{1.75in}
\addtolength{\parskip}{\baselineskip}
\setlength{\parindent}{0in}
\renewcommand{\baselinestretch}{1.5}
\newcommand{\term}{Winter 2018}

\singlespace


\begin{document}

\lecture{ 35 --- NoSQL }{\term}{Jeff Zarnett}

\section*{NoSQL}

NoSQL, well, originally ``non SQL'' or ``non relational'', is about data storage that isn't your typical relational database. Now it's ``Not Only SQL'', because some of the alternative databases can operate on SQL or SQL-like query languages (but that's sort of not what they are for). These days, NoSQL has become very popular and I sometimes have Fourth Year Design Project groups come to me and tell me they want to use NoSQL on a single user app that runs occasionally... Does that make sense?

If you ask them their motivations, such students will say something along the lines of, NoSQL is faster. As we will soon see, it can be but nothing comes for free. In any case, how much any particular app needs speed is an open question. Speed isn't everything. 

In reality, the right place to start is assuming you want a relational database, that is, a typical SQL based system. There are a lot of advantages to this and they tend to scale pretty well for a typical workload. We can scale them up and we can scale them out, as we have discussed in the distributed databases section. As we proceed through this discussion we'll see what the tradeoffs are and why we might be willing to give up the advantages of the relational database because we, more or less, have no choice.

\paragraph{Motivation.} The primary motivation for wanting to go to NoSQL would likely be scalability. That is to say, storing very large amounts of data, handling heavy workloads, and making data accessible to users wherever they are. When we say a lot of data we're talking about things like Twitter, Facebook, Uber, and other services that have astounding numbers of messages and millions if not billions of users (or, at least user accounts... there is a distinct possibility that a large fraction of Twitter and Facebook accounts are actually bots). 

An upside of NoSQL databases is that they call horizontally quite well. Unlike in SQL there is no need to do some magic to get a single server as big and fast as possible or to manually scale it across multiple servers. The MongoDB (one of the NoSQL vendors) documentation will tell you that one of the major advantages of the NoSQL approach is auto-sharding: no need to set up different database instances, get them to talk to each other, and the application probably needs be modified to work with multiple database locations. NoSQL, however, allows automatic sharding: they natively and automatically spread data across an arbitrary number of servers.  

Let's imagine that you do have an application that really does have huge amounts of users or operates over a lot of user data, or both (e.g., Amazon). Users do both reads and updates: they look at products on Amazon and products can be purchased and can also be sold out. Doing something like NoSQL might let you do it and have higher speed and higher availability, but there are costs. 

\paragraph{Tradeoffs.} What we want to do is ultimately limited by the CAP theorem, also known as Brewer's theorem, which says that there exists an iron triangle in distributed data stores. The iron triangle, you might remember from a discussion about project management, is ``fast, cheap, good: pick two''. The CAP theorem says that it is impossible to get more than two out of the following guarantees: Consistency, Availability, Partition Tolerance~\cite{brewercap}. To define these things:

\begin{itemize}
	\item \textbf{Consistency}: Every read receives either the most recent write, or an error.
	\item \textbf{Availability}: Every request gets a non-error response, with no guarantee it contains the most recent write.
	\item \textbf{Partition Tolerance}: The system can continue even if messages are lost between nodes.
\end{itemize}

The iron triangle is actually perhaps not the best analogy. It's not as if, at the start of the system or even during the design phase, you just choose what two items you want. No, actually, instead of 2 out of 3, it's choose either availability or consistency (maybe the descriptions gave you this hint). This is because there \textit{will} be network failures or at least delays, meaning there is partitioning, even if it is temporary. Then it's a question of what happens: if reads can happen before all notes are updates, we get availability; if systems require locking all nodes before allowing a read, we get consistency~\cite{bettercap}.

Let us consider two quick examples in the context of booking a flight. If we choose to prioritize consistency and there is network partitioning, then we would refuse to allow any updates like seat selection until such time as all servers are in agreement and the seat can be locked on all nodes. If we choose availability, then in the event of partition we allow the person to choose the seat anyway, which may be based on stale data. In that case, you might have let two people choose the same seat and there will be a need to sort this out later. For something light booking flights we might think it sensible to choose consistency over availability, because there is only one seat 32D on the airplane and it isn't exactly the same as any other seat. What if instead it was online shopping and you want to buy a book? One copy of the book is as good as any other, isn't it?

To give a real-life example of prioritizing availability over consistency, consider an online shopping scenario that really happened to me in November 2017. I was looking to buy a shirt and sweater, both of which were listed on the website as being available in the colours and sizes that I wanted. After suffering through the insufferable checkout process I received an e-mail telling me that I had successfully placed the order and the items would be shipped. Then, a while later I received another e-mail telling me they don't actually have the items and can't ship them and they're really sorry. And their website still showed the items as available. This is slightly embarrassing for the company, but it does happen.

For the most part, NoSQL databases do not provide ACID transactions. Instead, they may offer what is called \textit{eventual consistency}. This is to say that eventually, after some nontrivial period of time, the data will reach consistency. This is something you might have experienced (more or less) in applications like Facebook; if someone posts something you may not see it eventually as it could take some time for that thing to propagate its way through the network to you. Now, the problem is you might never catch up, because the content is constantly being updated. People are posting new pictures of their dinner at every moment.

Some of the NoSQL databases get speed by deleting some features that were very important throughout the course. There may be no joins. Really, no joins at all. And why would there be if there are no tables and therefore no good ways to relate two data elements together? If there are no joins then related data has to be stored together, namely in de-normalized schemas. This is exactly what it sounds like: it's a schema that is not in one of the normal forms that we discussed. On the contrary, data is duplicated and things that should probably be separated are kept together. Because it's fast, you see; joins are slow so let's not have joins!

Thing is, though, joins enforce consistency, so we might lose consistency in our database as a result of this. Duplicate data, or multiple entries of the same data that are all slightly different (Jane Doe, Jane R Doe...) any one of which you might get back in a particular query. For some things you may not care all that much -- if it is something like keeping track of how many megabytes of data the user has used this month then ``close enough is good enough''... Maybe.

Some NoSQL implementations do allow joins but they don't work quite like your typical SQL join with join attributes defined in a query... You can, if you want, write your own sort of join, where you do some sequential lookup: first look up record $x$ which contains a way to find related record $y$... This might be acceptable for finding an individual record, but to do so for 10~000 records is tedious. So perhaps you need to write your own join? At least we learned how those algorithms work... 

But then, why are you trying to reinvent the wheel? Relational database vendors have poured a lot of time and money and effort into optimizing every part of the database engine, including joins and fetches and the like. Even sometimes strange things like trying to time your code execution with the position of the read arm of the hard disk drive to micro-optimize a read or write operation~\cite{nosql}. 

Another way that NoSQL might work is it might not have transactions at all. Things happen, or don't, or might be halfway completed. 

So instead of working on the basis of relations with tables and keys and whatnot, how do NoSQL databases work? There's no simple, single answer to this because there are a lot of options. Remember, this is about things that are not standard relational databases. There are some options that support multiple modes, but we'll choose to focus on just two things for now: key-value databases, and document databases.

\paragraph{Key-Value Databases.} Surely at this point you are familiar with the idea of a key-value pair: in data structures and algorithms you learned about a Map (HashMap) for example that operates on the basis of \texttt{get( key )} and \texttt{put(key, value)}. 

Key-value pairs are a recommended way to store small amounts of data for things like Android applications. If you want user preferences in an app to be saved so it is remembered when you open the app again, this is one way to do it. And you can put arbitrary things in there, so you could save the configuration as XML or even just put the (serialized) data structure in there. As long as you know what it is and how to interpret it, you can get it out again. This is small-scale, obviously, but the idea has merit.

Now that idea is applied to a larger amount of data. Instead of defined tables with specific attributes, you have an arbitrary key and an arbitrary value. There may be no restrictions on the form of the key or the form of the value; they can just be  bytes of any length with no specific format, rules, or limits. This allows you to store anything you like... pdf documents, Java objects, flat text, comma-separated data... anything.

This scales well, since the only operations are get (read) and put (write) and there are no complexities related to searching, foreign keys, transaction management, et cetera. Eventual consistency can be achieved when every location, sooner or later, gets the update of the data. Put operations simply overwrite old data, and data elements (values) have no formal relationships to one another. You can create some ad-hoc ones by storing under one key a list of other keys, but this is informal.

\paragraph{Document Databases.} 

A document oriented store is somewhat more structured than the simple key-value pair. Instead of just treating the value in they key-value pair as an opaque series of bytes, we might analyze it and do something with it. Based on the structure of the value, as a document, it might be possible to extract some (meta)data about the document and then use that for something else (e.g., search). 

We could store lots of documents in the database in formats like XML, JSON, or binary data like PDF or Word documents. The documents themselves don't need to be formatted in any specific way, follow the same format or conform to any schema. And the content is completely arbitrary. If a document contains five XML tags, it has only those and there's no need to put nulls for XML elements that don't exist in that document. And, of course, no need to convert it from XML to an insert statement or convert the output of a select statement back into XML in the future. While there can be some tools that do that for you automatically so that you don't have to write the conversions manually, the work done is nonzero.

The basic operations are generally defined as CRUD~\cite{martin1983managing}: \textbf{C}reate (insert, put), \textbf{R}etrieval (query, search, get), \textbf{U}pdate (edit), \textbf{D}elete (remove). These can come under other names but this acronym is well known. These could be translated to get and put operations, but CRUD is a more friendly API that makes well to use cases.

Aside from the convenience of CRUD rather than get/put, the major advantage is the analysis that allows searching or perhaps better query performance. If we wanted to search for documents with a certain element, e.g., find all documents relating to a particular Tax ID, then having done the meta-analysis we could (maybe) find them relatively quickly if we have an index. If we are willing to have a bit more structure, we could allow differentiation of the data. Suppose the documents are e-mail and we have a bit of structure. Then if we want to search for ``abc@xyz.com'' we could find e-mails where that address corresponds to the recipient but not the sender.

A friend and database administrator referred at one point to a document database store as a ``Datengrab'', German for ``Data Grave''. It's where documents are placed when the die. If they ever come out it's nice, but admittedly in this particular use case, if the document happens to fall off a cliff it can always be regenerated from the source. This is, interestingly, a case where both things are used: there is a relational database that is used for most data, but documents are put into the non-relational document database. The choice is not binary.

\bibliographystyle{alphaurl}
\bibliography{356}


\end{document}
